\documentclass[a4paper]{article}
% LENGTHS---ADJUST THESE TO GET BETTER PAGE BREAKS
\newlength{\PARSKIP}\setlength{\PARSKIP}{1em}
\newlength{\ITMSEP}\setlength{\ITMSEP}{0.5em}
\newlength{\LASTUPDATESEP}\setlength{\LASTUPDATESEP}{1em}
\setlength{\footnotesep}{0.75\baselineskip}
\newlength{\BEFOREHEADER}\setlength{\BEFOREHEADER}{\PARSKIP}
\newlength{\AFTERHEADER}\setlength{\AFTERHEADER}{0.5\PARSKIP}
% PACKAGES
\usepackage{enumitem}
\usepackage[concrete]{fontsetup}
\usepackage[hang,perpage,symbol*]{footmisc}
\usepackage{microtype}
\usepackage[skip=\PARSKIP]{parskip}
\usepackage{showframe}
\usepackage{tabularx}
\usepackage{titlesec}
\usepackage{xurl}
\usepackage[colorlinks,allcolors=blue]{hyperref}
% VERTICAL SPACE BEFORE AND AFTER HEADERS
\titlespacing*{\subsection}{0pt}{\BEFOREHEADER}{\AFTERHEADER}
% ENUMERATION SETTINGS
\setlist[itemize]{leftmargin=*,align=left,labelsep=0.4cm,label=--}
\setlist[description]{leftmargin=*}
\setlist{parsep=\ITMSEP}
% ENVIRONMENT ITEMIZE WITH ADJUSTED VERTICAL SPACE BEFORE
\newenvironment{ITEMIZE}{\vspace{-\topsep}\begin{itemize}}{\end{itemize}}
% BEGIN DOCUMENT
\begin{document}
% SEC: CHARLES PROTEUS STEINMETZ
\section*{Charles Proteus Steinmetz}
% also possible to adjust the skip option to parskip package
% also possible to adjust the vertical enumeration spacing
\vspace{1em}
% LAST UPDATE
\vspace{\LASTUPDATESEP}
Last update:\@ \today{}%
\vspace{\LASTUPDATESEP}
% SUBSEC: BASIC INFO
\subsection*{Basic info}
%
\begin{tabularx}{\textwidth}{@{}lX}
     \textbf{Full name}  & Charles Proteus Steinmetz
  \\ \textbf{Birth year} & 1865
  \\ \textbf{Location}   & Schenectady, New York, United States
  \\ \textbf{Website}    & \url{https://en.wikipedia.org/wiki/Charles_Proteus_Steinmetz}
\end{tabularx}
% SUBSEC: EXPERIENCE
\subsection*{Experience\footnote{References provided upon request.}}
%% UNION COLLEGE
\textbf{1914--1923}                                             \\
Union College                                                   \\
\textit{Faculty member at the School of Electrical Engineering} \\
I worked pro bono as teacher and researcher.
%% PAR BREAK
\par
%% GEC
\textbf{1890--1923}      \\
General Electric Company \\
\textit{Engineer, Researcher, Consultant}
%% PAR BREAK
\par
%% FORD
\textbf{1920 (approximately)\footnote{%
  It was a long time ago. I do not remember the exact year.
}}                                                             \\
Ford (consultant)                                              \\
\textit{Chalk marker, Itemizer}                                \\
I worked mainly with ladders and chalk. I also itemized a bill.
%% PAR BREAK
\par
%% UNION COLLEGE
\textbf{1902--1913}                                    \\
Union College                                          \\
\textit{Chair of the School of Electrical Engineering} \\
In addition to my duties as chair, I worked pro bono as teacher and researcher.
%% PAR BREAK
\par
%% EICKEMEYER
\textbf{1889--1891}                           \\
Eickemeyer \& Osterheld Manufacturing Company \\
\textit{Engineer, Researcher}
% SUBSEC: EDUCATION
\subsection*{Education}
%% PHD STUDENT
\textbf{1883--1888\footnote{%
  Due to political reasons, I did not receive my degree until 1903, and when I
  did it was from Union University.%
}}                                                                            \\
University of Breslau                                                         \\
\textit{PhD}                                                                  \\
Thesis:\@ \textit{%
  On Involuntary Self-reciprocal Correspondences in Space Which Are Defined by a
  Three-Dimensional Linear System of Surfaces of the $n$th Order%
}
%% PAR BREAK
\par
%% ST. JOHN'S GYMNASIUM
\textbf{1873--1882}  \\
St. John's Gymnasium \\
\textit{Honors student}
% SUBSEC: MAIN COMPETENCES
\subsection*{Main competences}
\begin{ITEMIZE}
  \item%
  Chalk
  \item%
  Itemizing
  \item%
  Mathematics
  \item%
  Engineering
  \item%
  Research
\end{ITEMIZE}
% SUBSEC: LANGUAGES
\subsection*{Languages}
\begin{ITEMIZE}
  \item%
  German\\
  mother tongue
  \item%
  English\\
  fluent
  \item%
  Latin\\
  fluent
  \item%
  French\\
  fluent
  \item%
  Greek\\
  fluent
\end{ITEMIZE}
% END DOCUMENT
\end{document}
